\documentclass[letterpaper,11pt]{article}

\usepackage{booktabs}

\begin{document}
\paragraph{Inserting heterogeneities}:
Heterogeneities are inserted using parametric functions of the $(x,y,z)$
coordinates of the nodes or the center of the elements to which a
given \texttt{StGermain Variable Condition} is applied. As for regular
BC and ICs, the latest may be defined either as a constant value set
in the input file or as one of the function from the condition
function register.

The geometric parameters of a heterogeneity are defined in the XML
input file following the general syntax provided in\\
\texttt{Snac/plugins/heterogeneity/Hetero.xml.meta}.

Heterogeneities are inserted in the model in the same order you enter
them in the xml file. For instance, if the second heterogeneity
overlaps the space of the first heterogeneity for the same variable,
the variable will be set to the value provided by the variable
condition of the second heterogeneity.

Units are assumed to be the same with those for mesh geometry.
Defult value of \texttt{a\_shape} is set to 1.0 while defult values
of \texttt{b\_shape} to \texttt{e\_shape} are set to 0.0. Default
geometry is \texttt{FrontLimit}.

This means that by default the program will set all the point behind
$0$ to the value defined by the provided \texttt{Variable Condition}.
For normally set (i.e., $z_{min}\ge0$)model, nothing is going to happen.

There are currently 11 shape functions in this plugins:
\begin{enumerate}
\item Dyke: $2(Ax+By+Cz+D)^{2}/(A^2+B^2+C^2) \le t$
\item Sphere: $(x-Xc)^2+(y-Yc)^2+(z-Zc)^2 \le r^2$
\item Cylinder: $x=C_x z+p$; $y=C_y z+q$;
  $[(x-Az-p)^2+(y-Cz-q)^2+(C(x-p)+A(y-q))^2]/(A^2+C^2+1) \le r^2$
\item Cylinder\_V: $(x-Xc)^2+(z-Zc)^2 \le r^2$
\item Cylinder\_H: $(x-Az-B)^2 \le r^2$
\item UpperLimit: $y_p=-(Ax+Cz+D)/B \ge y$
\item LowerLimit: $y_p \le y$
\item RightLimit: $x_p=-(By+Cz+D)/A \ge x$
\item LeftLimit: $x_p \le x$
\item FrontLimit: $z_p=-(Ax+By+D)/C \ge z$
\item BackLimit: $z_p \le z$
\end{enumerate}

\begin{minipage}{1.0\linewidth}
\begin{tabular}{@{}cccccc@{}} \toprule
&a\_shape & b\_shape & c\_shape & d\_shape & e\_shape\\ \midrule
Dyke & A & B & C & D & thickness (r) \\ 
Sphere & $X_c$ & $Y_c$ & $Z_c$ & not used & radius (r) \\ 
Cylinder\footnote{will not work for vertical.} & $C_x$ & $C_y$ & $p$ & q & radius (r) \\ 
Cylinder\_V & $X_c$ & not used & $Z_c$ & not used & radius (r) \\ 
Cylinder\_H & $C_x$ & not used & $p$ & not used & radius (r) \\ 
UpperLimit\footnote{will not work for vertical.} & A & B & C & D & not used \\ 
LowerLimit & A & B & C & D & not used \\ 
RightLimit\footnote{will not work for $x=const$ vertical plane.} & A & B & C & D & not used \\ 
LeftLimit & A & B & C & D & not used \\ 
FrontLimit\footnote{will not work for $z=const$ vertical plane.} & A & B & C & D & not used \\ 
BackLimit & A & B & C & D & not used \\ \bottomrule
\end{tabular}
\end{minipage}

\paragraph{For programers}: To create a new shape, add the geometric
function in \texttt{TestCondFunc.c} and declare it in
\texttt{SnacHetero.h}. And call it within the function
\texttt{Is\_coord\_Inside} in \texttt{TestCondFunc.c} by creating a new
case in the \texttt{switch-case} block. Also define a name of
the geometry to be used in the XML input file in
\texttt{SnacHeterogeneity\_InitialConditions.c}. Don't forget to update
this documentation! ;-)


\end{document}
